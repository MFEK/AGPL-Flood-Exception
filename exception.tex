\documentclass[a5paper,14pt]{extreport}

\usepackage{fontspec}
\usepackage{microtype}
\usepackage{saveparinfo}
\usepackage{geometry}
\usepackage{setspace}
\usepackage{float}
\usepackage{draftwatermark}
\SetWatermarkText{DRAFT}
\usepackage{hyperref}

\geometry{a4paper, margin=1in}
\onehalfspacing

\title{Worldwide AGPL Exception for Modifications for the Purposes of Flood Control, Security, and CAPTCHA Software Integration}
\author{}
\date{}

\begin{document}

\maketitle

{\bfseries Notice: This is a {\em{draft}} of the MFEK AGPL security exception v1.0. Its text is not
yet final and may change significantly, pending community feedback. The repository will not be force
pushed after \texttt{1fb598d}, however, so if it already looks fine to you, feel free to use it.}

\section*{Preamble}

I, [Author's Name], the author of the program entitled [Program's Name] (the ``Program''), hereby
grant the following exception on a worldwide, irrevocable, perpetual and unlimited basis for the
said Program, which is licensed under the GNU Affero General Public License, version 3, or any later
version (the ``AGPL'') as published by the Free Software Foundation. This Exception is intended to
grant additional permissions to all users who incorporate flood control, security, or CAPTCHA
software (the ``Exempted Software'') into the AGPL-licensed software (the ``Program'').

By exercising the Exception, you agree to comply with the terms and conditions below:

\section*{1. Definitions:}
\begin{enumerate}
	\item[(a)] ``Flood Control'' refers to any software or measures designed to detect,
		mitigate, or prevent the excessive, repeated, or automated submission of data or
		requests, typically with the intent to disrupt the normal operation or functionality
		of a system or network.

	\item[(b)] ``Security'' refers to any software or measures designed to protect computer
		systems, networks, and their users from unauthorized access, exploitation, or harm,
		including but not limited to the confidentiality, integrity, and availability of
		information and resources.

	\item[(c)] ``CAPTCHA'' (Completely Automated Public Turing test to tell Computers and Humans
		Apart) refers to any software or measures designed to differentiate between human
		users and automated software, such as bots or scripts, by presenting tasks or
		challenges that are easily solvable by humans but difficult for machines to
		accomplish.

	\item[(d)] ``Exempted Software'' refers to any software written by end users that has as its
		primary purpose flood control, security, or functioning as a CAPTCHA.

	\item[(e)] ``Share-Alike Provisions'' refers to the sections of the AGPL that require
		derivative works to be distributed under the same license terms.
\end{enumerate}

\section*{2. Grant of Exception:}
\begin{enumerate}
	\item[(a)] If you incorporate the Exempted Software into the Program, you may, at your
		option, exclude the Exempted Software from the Share-Alike Provisions of the AGPL.

	\item[(b)] This Exception grants you the right to distribute, copy, and modify the Exempted
		Software and the Program, without the requirement to follow the Share-Alike
		Provisions concerning the Exempted Software.

	\item[(c)] This Exception does not grant you any additional rights or permissions beyond
		those described in this document.
\end{enumerate}

\section*{3. Conditions:}
\begin{enumerate}
	\item[(a)] The Exception applies only to the Exempted Software incorporated into the Program
		and does not extend to any other software or component of the Program.

	\item[(b)] You must include a copy of this Exception with the Program's source code and in
		any documentation accompanying the Program or the Exempted Software.

	\item[(c)] If you modify the Exempted Software, the Exception shall only apply to the
		modifications if they maintain the primary purpose of flood control, security, or
		functioning as a CAPTCHA.
\end{enumerate}

\section*{4. Termination:}
\begin{enumerate}
	\item[(a)] If you fail to comply with the terms and conditions of this Exception, it will
		automatically terminate and you will lose the rights granted by this Exception.

	\item[(b)] If you distribute the Exempted Software without this Exception, or if you remove
		this Exception from the Program, you will no longer have the rights granted by this
		Exception.
\end{enumerate}

\section*{5. Disclaimer of Warranty and Limitation of Liability:}
\begin{enumerate}
	\item[(a)] This Exception is provided ``as is'' without warranty of any kind, either
		expressed or implied, including, but not limited to, the implied warranties of
		merchantability and fitness for a particular purpose.

	\item[(b)] In no event shall the authors or copyright holders of the Exception be liable for
		any claim, damages, or other liability, whether in an action of contract, tort, or
		otherwise, arising from, out of, or in connection with the Exception or the use or
		other dealings in the Exception.
\end{enumerate}

\section*{6. Governing Law:}
\begin{enumerate}
	\item[(a)] This Exception shall be governed by and interpreted in accordance with the laws
		of the jurisdiction in which the original licensors of the AGPL-licensed software
		reside, without regard to its conflict of law provisions.

	\item[(b)] Any disputes arising from this Exception shall be subject to the exclusive
		jurisdiction of the courts of the jurisdiction in which the original licensors of
		the AGPL-licensed software reside.
\end{enumerate}

\newpage
\section*{7. How to Apply This Exception to Your Programs}

Attach the following notices to the program. It is safest to attach them to the start of each source
file to most effectively state the exclusion of warranty; and each file should have at least the
"copyright" line and a pointer to where the full notice is found.

~\saveparinfo{default}

\fbox{
	\small
	\begin{minipage}{.9\textwidth}
		\useparinfo{default}
		{
			{
				\bfseries
				\noindent <one line to give the program's name and a brief idea of what it does.>

				\noindent Copyright (C) <year>  <name of author>
			}

			\par{}This program is free software: you can redistribute it and/or modify it under
			the terms of the GNU Affero General Public License as published by the Free
			Software Foundation, either version 3 of the License, or (at your option) any
			later version (GNU AGPLv3+), supplemented by the additional permissions listed
			in the MFEK AGPL security exception v1.0, as published by the Modular Font
			Editor K Foundation Inc.. either version 1 of the exception, or (at your
			option) any later version (MFEK-Flood-Exception).

			\par{}This program is distributed in the hope that it will be useful, but WITHOUT ANY
			WARRANTY; without even the implied warranty of MERCHANTABILITY or FITNESS FOR A
			PARTICULAR PURPOSE.  See the GNU Affero General Public License and the MFEK
			AGPL security exception v1.0 for more details.

			\par{}You should have received a copy of the GNU Affero General Public License and
			the MFEK AGPL security exception v1.0 along with this program.  If not, see
			<\url{https://www.gnu.org/licenses/}> and <https://mfek.org/AGPL-Flood-Exception>.
		}
	\end{minipage}
}

~\\

In addition, be sure to provide a way for users to access the source code of your software,
especially if it can interact with users remotely through a computer network. This could be done,
for example, through a "Source" link in the interface of your web application.

You should also get your employer (if you work as a programmer) or your school, if any, to sign a
"copyright disclaimer" for the program, if necessary. For more information on this, and how to apply
and follow the GNU AGPL and this Exception, see <\url{https://www.gnu.org/licenses/}> and
<\url{https://mfek.org/AGPL-Flood-Exception}>.

\section*{Appendix A: About this document}
This document, titled ``Worldwide AGPL Exception for Modifications for the Purposes of Flood
Control, Security, and CAPTCHA Software Integration,'' is published by the Modular Font Editor K
Foundation Inc., a New Jersey Non-Profit Corporation. This is Version 1.0 of the document and can be
referred to in short as the ``MFEK AGPL security exception v1.0''. Programs licensed under the AGPL
and which use this exception may refer to themselves as
``\texttt{AGPLv$n$$+$MFEK-Flood-Exception-v1.0}'' programs.

To use this Exception in your own programs, simply include a copy of the Exception in your project's
source code and any documentation accompanying your program. Be sure to replace [Author's Name] and
[Program's Name] with your name and the name of your program, respectively.

For any inquiries or questions regarding this Exception, please contact:

\begin{itemize}
	\item Fredrick R. Brennan \\
\href{mailto:fred@mfek.org}{\ttfamily fred@mfek.org}
\end{itemize}

The most recent version can be found at \url{mfek.org/AGPL-Flood-Exception}.
\end{document}


